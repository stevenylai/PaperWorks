\section{The Composite Event Detection Problem}
\label{sec:system_model}
\begin{comment}
In this section, we formally define the problem of composite event detection.
\end{comment}
\subsection{System Model}
We consider the network as a graph \(G=(N, A)\) where each node represents a sensor node and each edge represents a communication link. For each \(a_n\in A\), it has a weight \(W_n\) associated with it.

The subscriber provides a finite set of event types \(E=\{e_1,e_2,\cdots\}\). For each \(e_n\in E\), the subscriber defines a set of attributes \(e_n\rightarrow attr_n\) which reflect certain real world phenomenon. 

The subscriber also provides a finite set of event relations \(R=\{r_1,r_2, \cdots\}\) where each \(r_n\in R\) represents the mapping of one or more sub-event types \(e_1, e_2\cdots \in E\) to a composite event type \(e_3\in E\), denoted as \(r_n(e_1, e_2, \cdots)=e_3\). One of the event type \(e_s\in E\) is subscribed by the subscriber. %Event types and their relations can be represented as a directed acyclic graph (DAG) as shown in Figure \ref{fig:eventdag}, where each node represents an event and each edge represents a relation between a sub-event and a composite event.

\begin{comment}
\begin{figure}
\centering
\figurecurrentwidth{eventdag}
\caption{Event DAG}
\label{fig:eventdag}
\end{figure}
\end{comment}

We have a set of primitive event types \(E_{primitive}\subseteq E\) such that \(r_n(e_1, e_2, \cdots)=e_n\) where \(e_n\in E_{primitive}\) and \(e_1, e_2, \cdots \in E\). For each primitive event of type \(e'_n\in E_{primitive}\), it will be detected by a node \(n_i\in N\). We use the message cost as the event detection cost for each event type \(e_n\in E\), denoted as \(cost(e_n)\).

\subsection{The General Problem Formulation}
Given:
\begin{itemize}
	\item A network \(G=(N, A)\)
	\item A set of event types \(E\) with relation \(R\)
	\item A cost function \(cost(e_n)\) for \(e_n\in E\)
\end{itemize}

Find:
\begin{itemize}
	\item For each event type \(e_n\in E\), when an event of this type is happens, find a subset of nodes \(V_n^r\subseteq V\) which are involved in detecting the event.
\end{itemize}

Objective:
\begin{itemize}
	\item Minimize the total energy consumption:
	\begin{displaymath}
	\sum_{i=i}^{n}cost(e_i)
	\end{displaymath}
\end{itemize}

\begin{theorem}
\label{thm:tableConstruction}
The composite event detection problem is NP-complete.
\end{theorem}
We may reduce this general composite event detection problem to Steiner tree problem since given a set of events, the minimum energy cost can be obtained if we connect these events through the Steiner nodes in the network. The detailed proof is omitted for the sake of brevity.

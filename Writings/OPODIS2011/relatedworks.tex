\section{Related Works}
\label{sec:relatedworks}
Existing works on pub/sub systems for WSN primarily consider the detection of primitive events where each event is usually treated separately and events don't have relations between each other. Low level naming \cite{lowlevelnaming} discusses a pub/sub system built on top of directed diffusion \cite{directeddiffusion}. The sink node will first broadcast interest and the source nodes will deliver the detected events through gradients and reinforced paths. Mires \cite{mires} is a pub/sub middleware for WSN. It makes use of the message-oriented communication paradigm provided by TinyOS \cite{nesc}. More recently, certain works such as \cite{lai:ted} have been proposed for composite event definition and detection. The primary focus of \cite{lai:ted} is on a composite event detection algorithm called TED which utilizes event type information for efficient detection. In addition to an event language, \cite{complexevent} also discusses how to reliably detect composite events in a pervasive environment. The focus of this paper is on a flexible middleware framework so that different event detection algorithms may be integrated and evaluated easily.

Apart from the pub/sub paradigm, there have been a lot of efforts on developing other type of programming abstractions for WSN including query-based approaches \cite{tinydb} VM-based approaches \cite{mate} tuple space-base approaches \cite{tinylime} neighbor-based approach \cite{hood} and mobile agent-based approaches \cite{agilla}.

We believe the pub/sub paradigm is suitable for many WSN applications because many of these applications are event-based in nature. However, more work needs to be done in order to support efficient composite event detection, especially how to support application-specific event detection mechanisms so that high energy efficiency can be achieved.

\subsection{SDN based Vehicular Network}

With the advancement of wireless communication and embedded technology, vehicles nowadays are becoming a powerful sensing, computing and communication platform. Vehicles can be connected to the Internet and ambient vehicles using wireless networks. However, challenges remain for developing practical and large-scale real world vehicular network applications. For example, the heterogeneity of wireless technology (3G/LTE, Wi-Fi, DSRC/IEEE 802.11p) used in vehicle communications has caused network fragmentation due to the difficulty in integration. In addition, the decentralized vehicle-to-vehicle (V2V) routing protocols are fragile to the highly dynamic vehicle mobility. The emerging software defined networking (SDN) technology provides new insights and alternative approaches to develop solutions of vehicular networks that can tackle the aforementioned problems. SDN has attracted great attention from both academia and industry in recent years. The separation of data plane and control plane and the employment of ��logically centralized�� management architecture can not only simplify the network management but also bring new services to the applications.

Currently, research on SDN mostly focuses on rethinking and redesigning wired network architectures, but not for wireless networks. In this project, being the first, we investigate the requirements and challenging issues in applying SDN to developing high-performance vehicular networks. More specifically, we design a framework to manage heterogeneous vehicular networks to improve the performance in multi-hop routing and inter-network flow switching and application-specific dataflow transmission. The framework is based on the abstraction and integration of the underlying heterogeneous wireless networks and provides a logically centralized way to greatly simplify the design of vehicular network management. As one of the key components, the framework consists of mechanisms and algorithms for collecting and maintaining the states of individual vehicles, which will be used by the SDN control plane to perform centralized optimization of routing and inter-network switching. In doing this, we will also address the challenging issues of network failures and recovery, vehicle mobility prediction and cooperative state updating. We will evaluate the effectiveness and performance of the proposed framework via simulation. We will also develop some demo applications based on the prototype.

This topic will make significant contributions to vehicular network and software-defined networks. This is the first project that utilizes SDN to solve data transmission problems for vehicular networks. The output of this project can help vehicular networks to be applied at large scale. Companies with large number of vehicles like logistic companies and taxi companies will benefit from it.

Our objective is 

\begin{enumerate}
  \item To study and identify the new requirements and issues of developing advanced vehicular applications.
  \item To propose and design the SDN based heterogeneous communication framework. The framework includes algorithms of inter-vehicular communication, inter-network integration and adaptive protocol utilization, with scalability and reliability.
  \item To evaluate the effectiveness and performance of the proposed mechanisms and algorithms through theoretical analysis and simulations.
  \item To build a test-bed and implement the proposed communication framework with mechanisms and algorithms, and develop example applications with demonstrations.
\end{enumerate}


\subsection{Walk-Safe: A pedestrian safety application for mobilephone uers}

Introduction
Recently, a research trend is focus on the effect of mobile phones on pedestrian safety due to wide usage of phones. Some studies demonstrate that mobile phone users exhibited more unsafe behavior than common people. For instance, the users are distracted and less aware of situation ahead, leading to falling and bumping injuries. Meanwhile, a number of research papers demonstrate some smart phone applications to enhance the pedestrian safety. Majority of them focus on utilizing the camera to help pedestrian be aware of unsafe condition. Though those applications can bring some benefit to users, they will bring great impact to battery-life due to utilizing high power consumption camera module. Thus, the energy drain issue has become a main drawback. To protect pedestrian and alleviate the battery drain issue, we have designed this safety application WalkSafe with a different approach. Instead of utilizing camera module, our system exploits an ultrasonic sensor attached into the mobile phone. Through analyzing the distance data collected from ultrasonic sensor, our system could help pedestrian avoid accident by notifying users peril ahead.
However, it presents several challenges in our work. For example, in order to alleviate the energy drain problem, a sophisticated power management policy should be proposed. On the other hand, a vibrating mobile phone produces noise and large variation in the readings from sensors due to the footfall, which should be filter out before being able to utilizing them. Furthermore, frequent meaningless disturbance tends to make users annoyed. Thus, we should figure out a way to reduce the disturbance to users. All these challenges are the barriers to build a robust pedestrian protect system under real-world conditions.
Consequently, we have set up the following requirements for our system: 1) Walker should be capable of real-time detection and notify users in a timely manner. 2) Walker should reduce the influence on mobile phone, including the battery consumption and system performance. 3) Walker should bring good user experience for users, reducing the meaningless disturbance time.

Objective
In this topic, we mainly have the following research objectives.
1.     Provide a safety application for pedestrian mobile phone users.

2.     Design a power management policy making the application more battery-friendly.

3.     Design a context-aware technology that minimizing the disturbance to user.

4.     Propose a method to reduce the noise in the data collected from sensors. 
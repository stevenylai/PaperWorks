\subsection{Privacy Measurement and Visualization in Participatory sensing}

The title of the problem is ��Privacy Measurement and Visualization in Participatory sensing��. The following is a detailed description.

The Problem

Given: there are N participants in a participatory sensing, each participant has m basic personal information items (e.g., nickname, age, address, phone number and etc).

Assume: each personal information item of the participants will be set privacy level by themselves.

Objective: calculate the privacy score for each participants based on their preference and participatory sensing application they join in.
The data may contain some personal information items so that reveal privacy information to others. We assume that all the personal information will be set privacy level when they join in the application.

The Motivation

It is important in terms of the following two aspects: (1) Most research work about privacy for different scenarios concentrate on privacy-preserving mechanism to protect personal information. However, the very fundamental thing is ignored that privacy has different meaning for everyone. (2) Since the privacy protection is critical, especially for participatory sensing application. The users would be loath to take part in the participatory sensing application without privacy-preserving. Finally, we believe this is the first work that measure and quantize the privacy for each user in participatory sensing application.

The Solution

The original idea behind Rasch Model is helping transform raw data from the human sciences into abstract, equalinterval scales. There are two attributes in the model: item difficulty and user ability, which are mapped to information sensitivity and participant��s attitude in the participatory sensing applications.

Currently, research on privacy protection mechanism focus on anonymization algorithm, ranging from k-anonymity , l-diversity  to t-closeness. However, these research are mostly stay at research level. We want to propose a usable privacy mechanism, more specifically, an App on the mobile to help users to manage their privacy.

The Results

The privacy measurement in participatory sensing has been proposed, which is based on sensitivity of users and anonymity of information, which represent users' preferences and the degree of anonymous of information. The experimental results illustrate that the pro-posed mechanism can provide effective measurement to users in participatory sensing. The App of privacy visualization is being developed and will be finish in one month.

Deliverables

The deliverables include:

\begin{enumerate}
  \item A paper about recommendation based on privacy measurement was submitted to TSC SI.
  \item A paper about privacy measurement will be submitted to IWQoS 2014.
  \item An Android App about privacy visualization will be developed and a paper will be submitted.
\end{enumerate}
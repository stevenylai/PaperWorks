
\subsection{S-Helmet}

Introduction:

Although the safety performance of the high-risk construction industry in Hong Kong has continued to improve, construction site accidents still accounted for nearly one-fifth of all the industrial accidents in Hong Kong. Safety policies and standards have proven to be less than completely reliable in many cases due to the complicated and dynamic environment in construction sites. ��Fall of person from height�� and ��Striking against or struck by moving or stationary objects such as construction hoist and tower crane�� are the top two killers in the construction industry.  It would be highly desirable to have a proactive safety management system able to continuously monitor workers�� locations and automatically issue warning alarm.

We propose to develop such a system for construction safety, called s-Helmet, based on the real-time localization technique we have designed.  In s-Helmet, wireless tags are attached to the safety helmets of workers and moving objects to track their locations in real time. When a worker is moving near danger zones (e.g. edge of the working platform, close to moving objects or electric discharge, etc.), the helmet will automatically issue an alarm to the worker as well as the construction site coordinators. We believe this system can effectively help to avoid the construction accidents and has great impact on the construction safety in Hong Kong. We summarize the functions of the S-Helmet.

\begin{enumerate}
  \item To prevent the occurrence of some dangers (e.g. fall of person from height and striking against a moving objects).
  \item To provide the real-time information for construction site management.
  \item To give real-time healthy information of the construction workers.
\end{enumerate}


Objective:

In this demo, we mainly have three research objectives:

\begin{enumerate}
  \item To design optimal MAC and routing protocol which is about to realize fast locali-zation of NanoLoc tags.
  \item To design a scheme to address the localization error caused by non-line of sight problem to the NanoLoc tags.
  \item To design energy efficient scheme to maximize the network lifetime of the tags.
\end{enumerate}

\subsection{A Conditional-Probability-Based Approach to Discovering Linear Correlation}

The correlation, in the era of Big Data, is now increasingly receiving favors form various data-driven applications and becoming a powerful approach for better understanding our surrounding lives. Because the correlation is only concerned with concurrent events regardless of relationships behind, such as causalities, we are able to discover complicated relationships without hypotheses by just looking at correlations. With more and more data being publicly available, taking advantage of the correlation has the power and is an effective approach to discovering interesting and valuable information. Linear correlation
is a kind of linear dependent relationship between variables, which can be used to precisly predict numeric values.

However, in the case of extremely large volume of variables, brute-force approach, which means examining each pair of variables by calculating correlation coefficient, is very computition-intensive. Mathetically, the necessity for pariwise variables to be linearly correlated is the existence of conditional probability between two variables.
So, we want to find an efficient way to estimate the conditional probability between variables.

In this topic, we mainly have the following research objectives:

\begin{itemize}
  \item To find an efficient way to estimate the conditional probability between variables.
  \item To design a algorithm to process large valume of correlation coefficient computation between variables.
  \item To implement a framework to discover linear correlation among large volume of variables.
\end{itemize}

\subsection{Correlation mining of user behaviour and social interaction}

Given the data collected from sensors and mobile devices, which is related to user behaviour and social interaction respectively,  the objective is to mine the correlation among social interaction and other users? data. The main motivation for this research is two-fold: firstly, mining the correlation among social interaction and other user information provides insight for psychology research, as the key factors that impact user?s social interaction can be unveiled. Secondly, it can also enable the new predication model for social interaction inference.

To mine the correlation among variable, a number of methods have been developed in different research community. Correlation analysis is quite popular to mine correlation in statistics area. Another way is to use mutual information to measure the correlation between two factors. The third method is called association rule, which can output the frequent item set and rules. Which method should be used depends on the attributes of data. Normally, numerical data fits the correlation analysis method. Discrete data is required for mutual information measurement, while categorial data correlation mining can be best handled by association rule.

The capability of mining the correlation among various kinds of data opens up many opportunities for business companies and governments. Great economical values can be created. For example, business component can leverage the correlation mining results to predict user?s behaviours, the trend of certain industries. Meanwhile, government is also significantly benefited. Air quality, human mobility and transportation situations can be predicted more accurately through correlation mining.

In this topic, we mainly have the following research objectives:

\begin{itemize}
  \item we will design a correlation mining framework which can process various kinds of dataset.
  \item we will propose a data transformation algorithm to transform the dataset into the required format of different mining methods.
  \item we will evaluate the effectiveness of methods from real-world dataset.
 \end{itemize}

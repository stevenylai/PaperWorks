\subsection{Resource Prediction}

The objective of this topic is to provision the resource for cloud services under the consideration of maximum the system utilization and users' cost. To solve the problem, two steps should be taken. First, we should predict the resource demand of cloud services accurately. Second, we should allocate appropriate resources to the given jobs. As is shown in Google workload trace published in 2011, the resource request is made manually. However, almost 60\% of the total jobs in the workload was killed by pro-grammers, which made a considerably CPU waste. What's more, some actual applications was delayed by resource contention. The average utilization of machines in the cluster is under 30\%. The real used resource is less than half of the resource request.

Consequently, resource prediction and resource allocation policies should be considered. Few works consider the resource provisioning problem on SaaS level, which actually is something of value. With a good resource prediction, one can save the cost for scheduling and migration. As I known, HP lab has done some work on the resource prediction for MapReduce services. However, their algorithm only considers the linear situation and ignores the communication between map and reduce tasks. Traditional resource prediction and scheduling policies only considers the homogeneous environment or workloads. How-ever, the workloads in the cloud computing environments has more diversity and heterogeneity. There are more challenges for resource provisioning for cloud services.

Three challenges exist to solve the problem. The first one is how to accurately modeling the workload. Maybe we can use the queuing theory to profile the arriving time, execution time, waiting time for a job.  The second one is how can be predict the resource demand as a vector other than a single value. We can try to solve the problem using newly multi-target prediction. The feasibility and accuracy should be tested and decided. The third one is how to determine the VM composition according to the resource demand. We can use integer optimization to solve the problem. I will develop a more accurate and light weighted model for resource prediction.

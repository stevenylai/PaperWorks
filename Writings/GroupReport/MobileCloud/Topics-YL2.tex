

\subsection{Service Placement}

The existing MCC model that enables the service provisioning directly from large centralized Internet data centers inherently incurs relatively high latency. Researchers point that the WAN latency could be as high as tens of milliseconds even when the users connect to the closest Internet data center. The latency may meet the requirement of most applications such as web browsing, but can yield bad usability/user experience for latency sensitive application such as high quality video streaming, mobile gaming, augmented reality and so on. To solve the problem, a lot of research works propose to add another abstraction tier between mobile users and centralized data centers.

We study the service placement problem in the 3-tier hierarchical system model (cloud-cloudlets-mobile users), in which the mobile users and cloudlets are geographically distributed. We have a set of services from the application provider to be placed on the cloudlets. The ideal case is that each user's demanding service is placed onto the cloudlet that is in the user's proximity. However, there are two possible cases that the users' demands are serviced from other cloudlets far away or the centralized cloud. The first case is that the users in proximity of the cloudlet may demand highly diverse services. However, placing all the demanded services onto the same cloudlet is not possible due to its storage capacity. Another case is we consider the compute capacity of cloudlets, and the load demands by mobile users from the cloudlet's proximity may exceed the cloudlet's capacity. Therefore, we face one basic problem which is to place a set of services onto cloudlets, and dispatch users' demands to the middle-tier cloudlets and the upper-tier cloud, such that the average latency of all the demands are minimized, while the capacity constraints of cloudlets are satisfied.

We further extend the service placement problem by considering the resource cost of the application provider. The resource cost of the application provider includes resource usages on the centralized cloud and distributed cloudlets. In particular, the cost on each cloudlet consists of the cost used to store the data associated with the placed services, and the compute cost which varies depending on the loads dispatched to the cloudlet. The cost on the cloud mainly includes the compute cost of the loads dispatched to the cloud. To ex-plain the resource cost of cloudlets, we envision two practical deployment settings of the cloudlets. One is that the cloudlets are deployed by wireless operators within the wireless access networks. Another is that cloud providers co-locate cloud resources in wireless ac-cess networks through co-location agreement with wireless operators. Throughout our pa-per, we name the party who provides cloudlet resources, e.g., the wireless operator in the former case or the cloud provider in the latter case, as the cloudlet provider. In real world, the cloudlets are from different cloudlet providers which may have various prices for the storage/compute resources. Besides the one-shot resource cost, we also consider additional cost incurred by the placement transition over two successive time slots. This is due to the fact that the services placed on cloudlets can be taken as replicas from the cloud. Placing new replicas on the cloudlet brings data transmission and thus bandwidths cost from the cloud to cloudlets. The extended service placement problem aims to balance the tradeoff between the operational cost and the average latency. The details of the problem description is as follows. 
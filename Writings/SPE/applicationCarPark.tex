\section{Application Scenario Three: Intelligent Car Park}
The final application that we are going to show in this paper is an intelligent car park \cite{tang:carpark}. It is related to ITS but there are a few distinct features with it so we will talk about it in a separate section.

\subsection{Requirement Analysis}
An intelligent car park application is different from other ITS applications in the following ways:
\begin{itemize}
\item Vehicle tracking is usually not needed. The application user is probably more interested in how many parking spaces have been occupied and how many are still available.
\item The sensor nodes probably shouldn't be deployed on the road. Instead, they should be deployed on each parking space to monitor its availability.
\item Different from other ITS applications, the message delay in intelligent car park is not really important. Message delay for a couple of seconds is probably acceptable. However, each message must be reliably delivered.
\item The environment for message transmission may be harsh and message lost may be common.
\end{itemize}

\begin{figure}
\centering
\subfloat[The sensor node]{\label{fig:carParkSensor1}\figurehalfwidth{carParkSensor1.JPG}}
\subfloat[The light sensor]{\label{fig:carParkSensor2}\figurehalfwidth{carParkSensor2.JPG}}
\caption{Car park sensor platform}
\label{fig:carParkSensor}
\end{figure}

\subsection{System Deployment}
We used the campus car park as our test bed. We deployed some micaz sensor nodes for the application. For simplicity, we use light sensor to detect the presence of a vehicle. For better communication, the sensor nodes are attached close to the ceiling instead of on the ground. The light sensor on each node is connected through an extended cable as shown in Figure \ref{fig:carParkSensor}. 

\begin{figure}
\centering
\figurecurrentwidth{carParkDeployment}
\caption{Car park sensor deployment}
\label{fig:carParkDeployment}
\end{figure}

The deployment of the individual sensor nodes is shown in Figure \ref{fig:carParkDeployment}. In such a system, the management is interested in the number of park spaces and the location of them \cite{tang:carpark}. Therefore, each parking space is monitored by one sensor node.

\subsection{Event Subscription}
The primitive events for such a system will be the availability of individual car park spaces. Based on the primitive event, if we want to get notified when the parking spaces near the exit become available, then we just need to define composite events which locate the spaces with certain IDs. Th event definitions are shown in Listing \ref{prog:carPark}. Here, the composite event takes two primitive events for parking space \(1\) and \(2\) which are close to the exit.

\begin{lstlisting}[caption=Event definition for a car park, label=prog:carPark]
Event ParkSpaceEvent {
	int id=System.id;
	int time=System.time;
	int light=System.light;
} where {
	light>THRESHOLD
}
Event CarParkEvent {
	int id=System.id;
} on {
	ParkSpaceEvent e1, e2;
} where {
	e1.id==1 ||
	e2.id==2 ||
	e1.time-e2.time<10
}\end{lstlisting}
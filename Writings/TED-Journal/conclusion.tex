\section{Conclusion}
\label{sec:conclusion}
In this paper, we presented TED, a type-based composite event detection. We formulate the problem of composite event detection and found it to be NP-complete. Therefore, TED uses a distributed randomized algorithm. We have given some important mathematical equations on how to adjust the parameters of TED in order to make the algorithm work in an energy efficient way. We evaluated the performance of TED through analysis, simulation and experiments and compared it with na\"{i}ve event detection where events are sent to the sink for detection. The results show TED can help to achieve high level energy efficiency without introducing too much delay.

Even though TED has good performance in detecting composite events, it may still have room for further improvement. For instance, current fusion point selection is done before the actual event detection. It might be possible if the fusion points are not only calculated before the event detection but also during the event detection process. We leave these as our future work.
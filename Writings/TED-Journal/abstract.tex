\begin{abstract}
Although there are several works on providing event-based services in pervasive environment or WSN, most of them have not considered composite event detection in an energy-efficient fashion. Composite events consist of multiple primitive events with temporal and spatial relations and are much more difficult to manage. Because of the resource constraints in WSN, existing event detection algorithms may not be suitable for WSN when energy efficiency is considered. In this paper, we propose TED (Type-based composite Event Detection), a distributed composite event detection algorithm. The essential idea of TED is type-based event fusion, where some sensor nodes are selected as fusion points. Then lower-level events will be fused on these fusion points for detection of higher-level composite events. Each composite event type is assigned to certain fusion point for detection so that the composite events may be detected in-network instead of at the sink. Event fusion with minimum energy cost is an NP-complete problem. We propose a centralized and a distributed randomized algorithm to solve the problem. In addition, as an important component in our algorithm, we designed a couple of schemes for deploying the fusion points for different application scenarios. We analyze the energy efficiency of TED according to these deployment schemes to show both its effectiveness and efficiency. By carrying out both simulation and real world experiments on TED, we show that TED can reduce the energy cost by 10-20\% in event-based WSN applications compared with na\"{i}ve event detection mechanism where the event relations are not considered.
\end{abstract}

\begin{keyword}
%% keywords here, in the form: keyword \sep keyword
Wireless Sensor Networks \sep Composite Event Detection
%% MSC codes here, in the form: \MSC code \sep code
%% or \MSC[2008] code \sep code (2000 is the default)

\end{keyword}
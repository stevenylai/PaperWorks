\section{Introduction}
Wireless sensor networks (WSNs), due to the advantage of low-cost and easy of deployment, have been widely used in many monitoring applications including battlefield surveillance, environmental monitoring and biological detection \cite{wsnhabitat}\cite{lo:ban}, etc. In these application, various practical problems, mainly associated with Quality of Service (QoS) such as sensor localization, coverage, event detection, target tracking, etc., have been studied extensively by researchers in computer science engineering. Through abstraction and assumptions, they transform these application-dependent problems into their own domain and solve them accordingly.

However, one potential application of WSNs that is far less investigated by computer science researchers is structural health monitoring (SHM).  The objective of SHM is to monitor the integrity of structures such as buildings, dams, bridges, and to detect and pinpoint the locations of any possible damage. Unlike other monitoring applications, detection of possible structure damage is not straightforward and requires significant amount of domain knowledge such as finite element model updating and damage indicator extraction \cite{farrar1997overview}. Moreover, many assumptions which were used to model the network associated problems, such as unit disk or convex sensing region, 0/1 local decision, or data aggregation by average, are unfortunately not realistic in SHM.  The required domain knowledge, along with the complexity of SHM, prohibits computer science researchers from investigating this application as intensely as others.

As a result, most research work so far in WSN-based SHM was done by researchers in civil engineering.  Using their knowledge in structure engineering, they have designed and developed many WSN-based SHM systems \cite{lynch2003embedment}\cite{nagayama2008structural}. However, based on our experience obtained from the previous collaborations with civil researchers, they generally concern whether the developed WSN-based SHM system can replicate the data delivery functionality of original wire-based counterpart and have less interest to embed in-network processing technology. Moreover, although they have solved many practical engineering problems, they still have difficulties to handle limitations of WSNs such as limited wireless bandwidth, limited communication range, and limited resources of wireless sensor nodes, etc.  When designing WSN-based SHM systems, civil engineers sometimes choose powerful wireless sensor nodes to accomplish work that could have been achieved by more cost-effective counterpart through system optimization. This leaves a large space to explore for computer science researchers.

In this paper, we demonstrate that computer science researchers can help fill this gap and significantly improve the performance of a WSN-based SHM system. We consider a fundamental problem in SHM: modal analysis, by which the vibrational characteristics of a structure are obtained.  These characteristics, called modal parameters, are basis for most of SHM algorithms and can also be used for vibration control and safety assessment. Traditional modal analysis is centralized which needs to stream all the measurement data back to a central unit. This method generally has high energy consumption and low scalability. In this paper, a cluster-based modal analysis approach is adopted. The basic idea of this approach is similar like the 'divide and conquer', where sensor nodes deployed on a structure are partitioned into clusters and modal analysis is carried out in each cluster. The resultant modal parameters of each cluster are then assembled together to obtain the modal parameters of the whole structure. In this approach, clustering is of great importance and should meet some extra requirements of modal analysis. Moreover, cluster size should be optimized to minimize the total energy consumption.

Our contribution in this paper is as follows:
\begin{enumerate}
\item	We show that design of a WSN-based SHM system is a multi-disciplinary area that the efforts from researchers in computer science engineering can significantly help improve system's usability and efficiency. 
\item We proposed a cluster-based modal analysis strategy. The clustering problem in this strategy is formulated and proven to be NP-complete.  Two centralized and one distributed algorithms are proposed.
\item	We evaluate our cluster-based modal analysis along with the clustering algorithms using the data from both simulation and real experiment.
\end{enumerate}

The remaining part of the paper is organized as follows. In section \ref{sec:relatedworks}, we present related works. Section \ref{sec:problem} describes the general framework for modal analysis, our architecture design and the clustering problem. The proposed algorithms for energy-efficient clustering are illustrated in section \ref{sec:solution}. Section \ref{sec:PerformanceEvaluation} includes a comprehensive evaluation of our cluster-based modal analysis using the simulation data and experiment data. Finally, section \ref{sec:conclusion} concludes the paper.
\section{Introduction}
\label{sec:introduction}
%DELETED: Distributed composite event detection \cite{jector} 
Distributed composite event detection in WSNs is needed in many applications such as health care \cite{lo:ban}, smart building \cite{lynch:shm} and intelligent transportation system \cite{klein:its}. For instance, in an intelligent transportation system, we may define traffic jam as an event when there are too many cars waiting on a road. Such kind of event may come from many sub-events such as the number of the vehicles on a road and the speed of the vehicles. Furthermore, the speed of the vehicles may come from the events of each individual vehicle. All these sub-events must satisfy certain spatial and temporal relations in order to indicate an occurrence of a composite event.

%DELETED: a special type of predicate detection \cite{gag:predicate}
Generally speaking, composite event detection in WSN can be treated as a special type of predicate detection. First, events in many WSN applications have strong temporal relations and have short life spans. For instance, object tracking usually involve events that occur as the object passes the network. After the object has exited the monitoring region, the primitive events related to that object will usually become useless. Second, events in many WSN have strong spatial relations. For example, to detect fire, a sensor node usually only need to compare the locally detected events with its neighbors' events. Third, data may be redundant and it is sufficient to detect one composite event instead of trying all the possible combinations and find out all composite events. This is due to the dense deployment of the sensor network. As an example, consider three sensors nodes \(n_1, n_2, n_3\) detecting over speeding vehicles by measuring the speed. If the speed is measured by two vehicle events and the three sensor nodes detect three events \(e_1, e_2, e_3\), then there will be \(C(3, 2)\) combinations. However, any event within the \(C(3, 2)\) combinations is sufficient to measure the vehicle's speed. Certain existing works \cite{lai:psware} have addressed the issue of defining and subscribing such kind of composite events for WSN. However, the problem of composite event detection has not been fully addressed in these works.

In comparison with that in active databases \cite{samos}, detecting composite events in WSNs poses more challenges because of the high dynamicity and resource constraints. Consequently, centralized approaches will usually suffer heavy traffic and the subsequent long delay, high energy cost and error rate. Therefore, a distributed solution seems more suitable.

In this paper, we address the problem of composite event detection for WSN. We propose CEDU (Composite Event Detection using event fUsion), a distributed algorithm for composite event detection in WSN. The main idea of CEDU is to leverage the event definition from application layer and use some sensor nodes called fusion points to detect composite events.

The contribution of this paper can be summarized as follows:
\begin{enumerate}
  \item We formulate the problem of composite event detection and find it to be NP-complete.
  \item We propose our distributed randomized algorithm, CEDU, for detecting composite events using event fusion. We also analyzed the effectiveness and efficiency of CEDU.
  \item We conducted some experiments based on some real world WSN applications on CEDU to show the performance of CEDU in detecting composite events.
\end{enumerate}
The rest of the paper is organized as follows: Section \ref{sec:relatedworks} reviews related works. Section \ref{sec:relatedworks} reviews related works. Section \ref{sec:system_model} presents our system model and problem formulation. Section \ref{sec:cedu} describes CEDU in details. Section \ref{sec:ceduanalysis} analyzes the effectiveness and efficiency of CEDU. Section \ref{sec:experiments} shows our experimental results. Section \ref{sec:conclusion} concludes the paper.
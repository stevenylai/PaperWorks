\section{Introduction}
\label{sec:introduction}
Wireless Sensor Network (WSN) is a popular topic in both research and industry due to its wide range of applications. In this section, we first illustrate some real-world WSN applications and discuss why it is necessary to create a middleware layer for these applications.

\subsection{Event-based Middleware for WSN}

\begin{figure}
\centering
\subfloat[Centralized]{\label{fig:room-a}\figurehalfwidth{room-a}}
\subfloat[Distributed]{\label{fig:room-b}\figurehalfwidth{room-b}}
\caption{Motivating application for indoor monitoring}
\label{fig:rooms}
\end{figure}

Event processing is important for WSNs application such as intelligent transportation system \cite{klein:its}, structural health monitoring \cite{lynch:shm} and healthcare \cite{lo:ban}. This is because the sensor nodes are used to detect the environmental changes and the applications need respond to these changes. Moreover, events may be primitive or composite. Primitive events can usually be detected by a single sensor node without cooperating with others. For instance, a primitive event may be defined as when the temperature exceeds certain threshold. While primitive events are used in many applications, they cannot describe the relations between different events. As a result, the applications need not just event processing but \emph{composite event processing} - the capability of distilling useful information from the relations among the data that are collected from different sensor nodes.

Because of the importance of event processing in WSN applications, we need a middleware for it for two reasons:
\begin{enumerate}
\item Since composite events are collaboratively processed by different sensor nodes in the network, they introduce extra complexity because of the resource constraints and unreliable communication. As a results, we need a middleware for event processing.
\item It will be easier to develop WSN applications if the middleware can provide a simple and consistent way for defining events.
\end{enumerate}

Composite event processing is challenging. For instance, in an intelligent transportation system, the traffic jam event may be defined as too many cars waiting on a road. This event, however, may not be directly detected by a single sensor node. The traffic jam event may come from many sub-events such as the number of the vehicles on a road and the speed of the vehicles. The speed of the vehicles may, in turn, be collaboratively measured by different sensor nodes on a same road. Once a traffic jam event has been detected, we may need to inform other vehicles or the traffic light controller so that the corresponding adjustment may be made to improve the efficiency of the transportation system. All the processing need to be done in a resource constrained environment with unreliable wireless communication. Furthermore, very few practical applications can rely purely on primitive events. That is why a composite event-based middleware is particular important for WSN applications.

Event-based middleware can provide a common event-based programming model for different applications. Even though different WSN applications may have different requirements, their common event-based nature makes it possible to let the application developer define those requirements through events. We show how this can be possible through several examples. First, consider a monitoring application shown in Figure \ref{fig:rooms}. The composite event consists of two sub-events to be detected in two different rooms, Room A and Room B. It occurs when the temperature in room A rises to a certain threshold and after 5 minutes, the temperature in room B also reaches that threshold. To define such a scenario, the user will probably set up some composites events through some logical operators such as: \(roomA.temp \geq 30, roomB.temp \geq 30, timeB - timeA \geq 5\) Second, we consider an ITS application scenario. In such an application, as shown in Figure \ref{fig:its}, the sensor nodes are placed along the road to detect the presence of the vehicles. These events are gathered for various purposes such as collision avoidance, traffic light control and over-speed detection. Similarly, those scenarios may be defined through some operators. For instance, if we divide the distance by time, then we can obtain the speed of a car. If the speed is greater than a threshold, then it is considered as over-speeding. As we can see from these examples, it is possible for the application users to make use of operators and events to express the application requirements for different applications.

\subsection{Desirable Features of Event-based Middleware}

\begin{figure}
\centering
\subfloat[Centralized]{\label{fig:its-centralized}\figurehalfwidth{its-centralized}}
\subfloat[Distributed]{\label{fig:its-distributed}\figurehalfwidth{its-distributed}}
\caption{Motivating application for ITS}
\label{fig:its}
\end{figure}

In order to support the application development in WSN, the event-based middleware should have the following features:
\begin{itemize}
\item \emph{High level support for composite events}: application users may interact with the underlying event services through the high level abstraction without having to worry about the underlying event processing mechanisms.
\item \emph{Flexible architecture for event processing}: it is possible to implement customized event processing mechanisms without affecting the operations of high level users
\end{itemize}

As discussed in the previous subsection, it is possible to define events for different WSN applications in a common way from the application programmer's point of view. However, the underlying mechanisms for processing composite events can be very different among different applications. Different from centralized approaches, where all the data can be gathered for processing the events, in WSN, sensor nodes usually do not have the global knowledge of all the events in the network. Instead, many events are detected locally to save communication costs. Moreover, different applications may have different requirements for event detection. Some applications may want energy saving while others may want higher reliability. We illustrate this through our previous application examples.

For the indoor application, a centralized approach to this application is shown in Figure \ref{fig:room-a}. It gathers all the temperature data to the sink and perform the event detection. However, this approach will quickly deplete the energy on the sensor nodes. As shown in Figure \ref{fig:room-b}, a more economic approach is to select a sensor node for each room as the sub-event detector. The detector will gather the temperature readings from each room locally and try to detect the sub-event first. Then, only after the sub-event has been detected, the data will be transmitted to the sink node. By doing so, we effectively avoid transmitting all the raw data to the sink. Similarly, detection of these events can be centralized or distributed as shown in Figure \ref{fig:its-centralized} and \ref{fig:its-distributed}. However, different from the indoor monitoring application, in ITS, it might be better if the sensor nodes detect the vehicles along the road instead of forwarding the data to the center of the road. 

Consequently, the event-based middleware can help by hiding the underlying event processing details from the application developers and by doing this, modifications in the event processing mechanisms will not affect the high level applications. 

\subsection{Our Contributions}
In this paper, we propose a middleware framework called PSWare. PSWare uses a flexible architecture where different event processing algorithms can be easily integrated. The contribution of the paper can be summarized as follows:
\begin{itemize}
\item We described the design of PSWare. PSWare features a flexible architecture. Different application domains can easily implement event processing mechanisms on top of PSWare.
\item We developed several real WSN applications on top of PSWare. The applications are from different domains such as indoor temperature monitoring, intelligent transportation system and structural health monitoring.
\item We performed experiments on PSWare to demonstrate its effectiveness in helping developing event-based applications in WSN.
\end{itemize}

A lot of middleware work for WSN has been proposed. Existing approaches event-based middleware for WSN mostly include query-based approach \cite{tinydb} and publish/subscribe approach \cite{complexevent}. However, these works either cannot work well for composite event-based systems or are not flexible enough to allow implementation of customized event detection algorithms. 

%In comparison with that in distributed systems, detecting composite events in WSN poses more challenges because of the high network dynamics and resource constraints. First, event relations may need to be taken into consideration when detecting. For example, 

%There are already some works that have proposed event detection algorithms \cite{lai:ted} for WSN. However, in order to make these algorithms work for real world WSN applications, more work needs to be done. One approach is to implement application-specific algorithms. In this approach, it is easier to improve the event detection efficiency by taking advantage of application-specific information. However, we may have to re-invent the wheel if similar algorithms are implemented in different applications. A better approach is to have a generic software layer for event detection and applications will be built on top of this. This can effectively avoid re-inventing the wheel but the software must be flexible enough so that it is capable of supporting a wide range of applications.

The rest of the paper is organized as follows: Section \ref{sec:background} gives the background and related works for PSWare. Section \ref{sec:design} describes the middleware design in details. We then describe how to implement applications based on PSWare. Section \ref{sec:pswareImpl} shows how to use PSware to implement applications such as indoor monitoring, intelligent transportation systems and structural health monitoring. Section \ref{sec:its} shows how to use PSWare to develop applications for intelligent transportation systems. Such applications are more time-critical. Section \ref{sec:experiments} shows our experimental results. Section \ref{sec:conclusion} concludes the paper.
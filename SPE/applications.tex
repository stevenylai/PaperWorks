\section{Application Scenarios}
\label{sec:pswareImpl}
In previous sections, we have introduced PSWare and how it can be customized to meet different application requirements. In this section, we demonstrate how PSWare can help in developing practical WNS applications through three application case studies: the indoor monitoring, the car park and the intelligent transportation system.

The key features of the three applications are as follows:
\begin{itemize}
\item Energy efficiency in an indoor monitoring system is crucial because the network needs to operate for a long period of time and it is not necessary to gather all the raw data.
\item Reliability is important in a car park system because the management system needs to know the time when each car enters and leaves the car park so as to calculate the fee.
\item Event detection and delivery mechanism may be tailor-made for an ITS system because transportation network has certain special 
\end{itemize}


\subsection{Applications Overview}
Our first application case study is an indoor monitoring system. Such application domain has many applications including air-conditioning, fire alarm and office space light adjustment. Such system is a good application for using composite events because of the following characteristics:
\begin{itemize}
\item There may be a large amount of raw data since the data may be gathered from a lot of sources.
\item The users may not be interested in all the raw data. Instead, they may be more interested in change of the data. This is because the events such as fire or temperature change will usually trigger the change of the sensory data. 
\item Primitive event detection may not be sufficient. For instance, a fire alarm may be characterized by a combination of different types of events such as temperature and light.
\end{itemize}

Our second application case study is ITS. This application also has a lot of applications including collision avoidance, traffic light control, automatic road enforcement and electronic toll. Composite event detection is very common in these applications because one of the common requirements for these applications is to detect the speed of vehicle. However, different from the indoor temperature and light monitoring, the ITS applications have the following characteristics:
\begin{itemize}
\item The magnetic sensor used in ITS application will consume much more energy so the energy consumption from sensing may not be negligible. This will affect the event detection method for ITS.
\item The sensor nodes are deployed and lined up on each road and will detect each passing vehicle one by one. Such special deployment may also affect the event detection method.
\item Events may need to be delivered to mobile sinks instead of a fixed point.
\item Different types of events may require different event delivery methods. Urgent events such as accident may need to be disseminated in a real time fashion while events such as road condition and weather information may be delivered in a delay tolerant fashion.
\end{itemize}

Our final application is an intelligent car park application. Though this applications may sometimes be considered as one of the applications in ITS, it is different from other ITS applications in the following ways:
\begin{itemize}
\item The detected events usually need to be delivered to a central management center instead of to mobile vehicles.
\item The deployment of the sensor nodes is different. In a car park system, it is usually not necessary to deploy the sensor nodes on the roads. Instead, they are usually deployed in individual parking space to keep track of availability of them.
\item Different from other ITS applications, the event delay in intelligent car park is not as important. Message delay for a couple of seconds is probably acceptable.
\end{itemize}

On the other hand, a car park system is also different from our first indoor application in the sense that:
\begin{itemize}
\item Each message must be reliably delivered so that the management system can correctly record the entrance and exit time and calculate the parking fee.
\item The environment for message transmission may be harsh because cars are usually made of metal. Therefore, message lost may be common.
\end{itemize}
To meet the reliability requirement, we need to implement special event detection and delivery mechanisms so that the events can be reliably detected and delivered in such an unreliable environment.

\chapter{Introduction}
\section{Overview}
\label{sec:introduction}
Development in wireless communication and electronics has made it possible to create low-cost, low-power wireless sensor nodes. Each sensor node usually contains a wireless transceiver, a micro processing unit and a number of sensors. The sensor nodes can collect data and do some simple processing locally, and can communicate with each other to form an ad hoc wireless sensor network (WSN) \cite{aky:survey}. A WSN is usually self-organized and self-maintained without any human intervention. Wireless sensor networks have been used in various application areas such as smart building \cite{lynch:shm}, wild environment monitoring \cite{wsnhabitat}, intelligent transportation systems \cite{klein:its}, battle surveillance \cite{wsntracking} and healthcare \cite{lo:ban}. 

While WSN has a wide range of applications, programming sensor networks is a non-trivial task since it requires programming individual sensor nodes, using low-level programming languages, interfacing to the hardware and the network which it is only supported by primitive operating system abstractions \cite{programmingparadigms}. In this research, we propose PSWare, a publish/subscribe (pub/sub) middleware for WSN which eases the development of WSN applications. Our middleware uses a pub/sub programming paradigm for the application programmer to subscribe application specific events. It provides an Event Definition Language (EDL) which allows the users to define composite events. With the help of PSWare, we show it is much easier to develop some typical applications for WSN.

\subsection{Research Motivation}
Distributed composite event detection \cite{jector} in WSNs is needed in many applications such as healthcare \cite{lo:ban}, smart building \cite{lynch:shm}, and intelligent transportation system \cite{klein:its}. For instance, in an intelligent transportation system, we may define traffic jam as an event when there are too many cars waiting on a road. Such kind of event may come from many sub-events such as the number of the vehicles on a road and the speed of the vehicles. Furthermore, the speed of the vehicles may come from the events of each individual vehicle. All these sub-events must satisfy certain spatial and temporal relations in order to indicate an occurrence of a composite event. In these applications, due to the event complexity and large network scalability, the centralized event detection approach will suffer heavy traffic and subsequently, long delay, high energy cost and error rate.

Because of the common requirements and challenges for composite event subscription and detection, it is more desirable to have a generic middleware layer to handle composite events instead of reinventing the wheel and implementing application-specific event processing mechanisms. In addition, programming in WSN is still considered challenging even though there are a lot of programming support for programming WSN \cite{nesc}. The programmers need to deal with low level issues such as scheduling, concurrency and resource management. 

\subsection{Research Issues}
In this research, we address the essential issues of designing PSWare. On the highest level, an event description language is provided to allow users to describe composite event relationships. On the lowest level, a runtime environment is necessary on each sensor nodes so that they could execute bytecodes compiled from the high-level EDL. More specifically, we need to address the following research/engineering issues to make our middleware work.

\begin{itemize}
\item	\emph{Event definition language}: we need to provide an event description language which is powerful yet easy to use.
\item	\emph{Event definition language compiler}: a compiler is essential to translate the programs written in our high-level language into a low-level language understandable by the sensor nodes. The compiler must be smart enough to extract the semantic meanings from the program and do some optimization.
\item	\emph{Runtime environment support}: the middleware running on each sensor node should provide a runtime environment to execute the compiled programs.
\item	\emph{Subscription propagation}: after the program written in EDL is compiled, the subscription should be intelligently disseminated into the network. If the subscription is too big, it may need to be divided into small ones. Furthermore, only related sensor nodes need to be updated.
\item	\emph{Event detection}: we need to design efficient protocol for the sensor nodes to cooperate with each other to detect composite events.
\item	\emph{Event delivery}: after the subscribed the events have been detected, we need energy-efficient routing protocols so that the detected events will be delivered at the minimum cost.
\end{itemize}
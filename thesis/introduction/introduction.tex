\chapter{Introduction}
\section{Overview}
\label{sec:introduction}
Development in wireless communication and electronics has made it possible to create low-cost, low-power wireless sensor nodes. Each sensor node usually contains a wireless transceiver, a micro processing unit and a number of sensors. The sensor nodes can collect data and do some simple processing locally, and can communicate with each other to form an ad hoc wireless sensor network (WSN) \cite{aky:survey}. A WSN is usually self-organized and self-maintained without any human intervention. Wireless sensor networks have been used in various application areas such as smart building \cite{lynch:shm}, wild environment monitoring \cite{wsnhabitat}, intelligent transportation systems \cite{klein:its}, battle surveillance \cite{wsntracking} and healthcare \cite{lo:ban}. 

While WSN has a wide range of applications, programming sensor networks is a challenging because different from programming in the traditional environment. WSN imposes a lot of constraints such as limited computational power, less memory and unreliable communication. Application developers need to not only understand the requirements for the specific application domain but also take into consideration of the characteristics of WSN. In this research work, we propose PSWare, a publish/subscribe (pub/sub) middleware for WSN which eases the development of WSN applications. Our middleware uses a pub/sub programming paradigm for the application programmer to subscribe application specific events. It provides an easy-to-use Event Definition Language (EDL) to allow the application developers to define composite events while uses a flexible architecture so that different domain-specific event processing algorithms can be easily integrated into the middleware.

\subsection{Motivation}
Despite the large variety of WSN applications, many of them are essentially event-based. In applications such as intelligent transportation systems \cite{klein:its}, smart buildings \cite{lynch:shm} and healthcare \cite{lo:ban}, the events sensor nodes detect events which reflect the environmental changes and the systems respond to these events accordingly. Events may be primitive or composite. Primitive events (e.g. when the temperature exceeds certain threshold) can be detected by a single sensor node without having to cooperate with others. On the other hand, \emph{composite events} consist of multiple primitive events. They reflect a serial of environmental changes with spatial and temporal relations among them and must be detected collaboratively by different sensor nodes.

Because of the common requirements and challenges for composite event subscription and detection, it is more desirable to have a generic middleware layer to handle composite events instead of reinventing the wheel and implementing application-specific event processing mechanisms. In addition, the middleware should be flexible enough so that different event processing algorithms for different application domains can be easily integrated. In summary, the middleware should achieve the following design goals:
\begin{enumerate}
\item \emph{Event abstraction}: since composite events are collaboratively processed by different sensor nodes in the network, they introduce extra complexity because of the resource constraints and unreliable communication. A middleware framework providing high level event abstraction is needed to ease the application development.
\item \emph{Re-usability and flexibility}: different applications may share certain common modules during event processing. They may also have other different modules. A middleware framework can help so that common modules may be used and different modules may be replaced without affecting others.
\end{enumerate}

\subsection{Issues}
In this research, we address the essential issues of designing and developing PSWare. On the highest level, an event description language is provided to allow users to describe composite event relationships. On the lowest level, a runtime environment is necessary on each sensor nodes so that they could understand and execute the event processing algorithms translated from the high-level EDL. More specifically, we need to address the following research/engineering issues to make our middleware work.

\begin{itemize}
\item	\emph{Event definition language}: we need to provide an event description language which is powerful yet easy to use.
\item	\emph{Event definition language compiler}: a compiler is essential to translate the programs written in our high-level language into a low-level language understandable by the sensor nodes. The compiler must be smart enough to extract the semantic meanings from the program and do some optimization.
\item	\emph{Runtime environment support}: the middleware running on each sensor node should provide a runtime environment to execute the compiled programs.
\item	\emph{Subscription propagation}: after the program written in EDL is compiled, the subscription should be intelligently disseminated into the network. If the subscription is too big, it may need to be divided into small ones. Furthermore, only related sensor nodes need to be updated.
\item	\emph{Event detection}: we need to design efficient protocol for the sensor nodes to cooperate with each other to detect composite events.
\item	\emph{Event delivery}: after the subscribed the events have been detected, we need energy-efficient routing protocols so that the detected events will be delivered at the minimum cost.
\end{itemize}
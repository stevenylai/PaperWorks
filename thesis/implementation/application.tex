\chapter{System Implementation}
\label{chapter:implementation}
Since the goal of using middleware is to support application development. In this chapter we illustrate, how PSWare can be implemented for different applications.

\begin{algorithm}
\begin{algorithmic}[1]
\REQUIRE \(v_n\rightarrow msg_r\)
	\FOR {each entry \(t'\) in \(msg_r\)}
		\IF {\NOT exists \(t'\rightarrow fid_n\) in \(table_r\)}
			\STATE \(addTo(table_r, t')\)
		\ENDIF
		\FOR {each entry \(t\) in \(table_r\)}
			\IF {\(t\rightarrow fid_n = t'\rightarrow f_n\)}
				\IF {\(t'\rightarrow hop_n < t\rightarrow hop_n\)}
					\STATE \(t\rightarrow hop_n \gets t\rightarrow hop_n+1\)
					\STATE \(t\rightarrow parent_n \gets v_n\)
				\ENDIF
			\ENDIF
		\ENDFOR
	\ENDFOR
	\IF {self is fusion point \AND \NOT exists \(self\rightarrow id\) in \(table_r\)}
		\STATE \(addTo(table_r, (self\rightarrow id, 0, self\rightarrow id))\)
	\ENDIF
	\STATE \(msg_r \gets table_r\)
	\STATE \(periodically\_broadcast(msg_r)\)
\end{algorithmic}
\caption{\(table_r\) construction}
\label{algo:table_r}
\end{algorithm}

\section{Implementation of PSWare with TED}
We discuss how PSWare can be used to easily support different event detection algorithms. In this section, we first study how PSWare can help in composite event processing in general. Since a number of algorithms have been proposed for composite event detection in WSN. In particular, \cite{lai:ted} describes a composite event detection algorithm named TED that emphasizes on energy saving. We use TED as an example for implementation.

We first briefly review the idea of TED. TED is a distributed type-based composite event detection algorithm. It uses certain amount of pre-selected nodes as so called event fusion points which are responsible for detecting composite events. The algorithm will first use mathematical methods to calculate the near optimal number of the fusion points. These fusion points will then be deployed in the network randomly or evenly. Then based on these fusion points, the sensor nodes will update the routing and the events will be forwarded to the fusion points so that composite events may be detected with less cost.

\subsection{Data Structures}

\begin{algorithm}
\begin{algorithmic}[1]
\REQUIRE \(v'_k\rightarrow msg_m\)
	\FOR {each entry \(t'\) in \(msg_m\)}
		\IF {\NOT exists \(t'\rightarrow e_n\) in \(table_e\)}
			\STATE \(addTo(table_e, (t'\rightarrow e_n, 1, v'_k, t'\rightarrow cost_n^i+table_r\rightarrow hop_k))\)
		\ENDIF
		\FOR {each entry \(t\) in \(table_e\)}
			\IF {\(t\rightarrow e_n = t'\rightarrow e_n\)}
				\IF {\(t'\rightarrow cost_n^i+table_r\rightarrow hop_k < t\rightarrow cost_n\)}
					\STATE \(t\rightarrow cost_n \gets t'\rightarrow cost_n^i+table_r\rightarrow hop_k\)
					\STATE \(t\rightarrow fusion_n \gets v'_k\)
				\ENDIF
			\ENDIF
		\ENDFOR
	\ENDFOR
	\IF {self is fusion point}
		\STATE \(msg_m \gets table_m\)
		\STATE \(periodically\_broadcast(msg_m)\)
	\ENDIF
\end{algorithmic}
\caption{\(table_e\) construction}
\label{algo:table_e}
\end{algorithm}

We have a set of nodes \(V'\subseteq V\) selected as event fusion points. Each individual sensor nodes will maintain the following table \(table_r\) for all \(v'_n\in V'\)
\begin{itemize}
\item Fusion point ID (\(fid_n\)): the node ID of the fusion point:
\item	Hop count (\(hop_n\)): the number of hops to reach the fusion point
\item	Parent (\(parent_n\)): the next hop to that fusion point\\
\end{itemize}
Each sensor node will also maintain an event table \(table_e\) containing the information for each event type \(e_n\in E\) which includes the following fields:
\begin{itemize}
\item Event type ID (\(e_n\)): the ID which is assigned to each event type
\item Fusion point for the event (\(fusion_n\)): the fusion point at which the event is mostly likely to be detected at the lowest cost.
\item Fusion cost (\(cost_n\)): the fusion cost for event type \(e_n\)\\
\end{itemize}
In addition to the above tables, each fusion point \(v'\in V'\) will maintain another table \(table_m\) for the purpose of matching events. \(table_m\) contains the following fields:
\begin{itemize}
\item Event type ID (\(e_n\)): the ID of the event type
\item Event instance ID (\(i\)): the \(i\)th event instance of event type \(e_n\) (we use \(e_n^i\) to denote such an instance of event)
\item Source node (\(v_n^i\)): the node which forwarded \(e_n^i\) to the fusion point
\item Event timestamp (\(t_n^i\)): the timestamp when the event \(e_n^i\) is detected
\item Detection cost (\(cost_n^i\)): the cost for detecting event \(e_n^i\)
\end{itemize}

\subsection{Table Construction}

\begin{algorithm}
\begin{algorithmic}[1]
\REQUIRE detected event \(e_n^i\)
	\IF {\(table_e\rightarrow fusion_n=null\)}
		\STATE \(event\_forward(e_n^i)\)
	\ELSE
		\STATE \(send(e_n^i, table_e\rightarrow fusion_n)\)
	\ENDIF
\end{algorithmic}
\caption{Event forwarding}
\label{algo:eventforwarding}
\end{algorithm}

Each node \(v_n\) periodically broadcasts messages \(msg_r\) which is its \(table_r\). If the node itself is a fusion point, then it will add itself in \(table_r\) and broadcast the message. The procedure is shown in Algorithm \ref{algo:table_r}.

In addition to \(msg_r\), each \(v'_k\in V'\) will periodically advertise its \(table_m\) by broadcasting \(msg_m\) so that other sensor nodes can construct their \(table_e\) with Algorithm \ref{algo:table_e}.

The construction of \(table_m\) will take place when the event instance \(e_n^i\) is detected and forwarded to a fusion point \(v'_n\). We will discuss how forwarding could be done in the next subsection.

\subsection{Event Forwarding and Matching}

\begin{algorithm}
\begin{algorithmic}[1]
\REQUIRE \(e_n^i\) detected by \(v_k\) with cost: \(cost_n^i\)
	\STATE \(addTo(table_m, (e_n, e_n^i, v_k, now(), cost_n^i))\)
	\FOR {each \(e_j\) in \(E\)}
		\IF {\(\exists r\in R\) \AND \(r=e_{comp}=comp(e_n, e_j)\)}
			\FOR {each \(e_j^k\) in \(table_m\)}
				\IF {\(comp(e_n^i, e_j^k)=true\)}
					\STATE \(addTo(table_m, (e_{comp}, e_{comp}^i, self, now(), cost_n^i+cost_j^k))\)
					\STATE \(detected(e_{comp})\)
				\ENDIF
			\ENDFOR
		\ENDIF
	\ENDFOR
	\STATE \(event\_forward(e_n^i)\)
\end{algorithmic}
\caption{Event matching}
\label{algo:eventmatching}
\end{algorithm}

When an event \(e_n^i\) is detected at node \(v_k\), node will use \(table_r\) and \(table_e\) to decide how to forward the detected event to the fusion points so that higher level events can be detected. The pseudo code is shown in Algorithm \ref{algo:eventforwarding}.

In case the fusion point for event type \(e_n\) has not been decided, the node will forward the event to some of its closet fusion points using the \(event\_forward\) function according to TED. Upon the reception of \(e_n^i\) from \(v_k\), the fusion point will first update its own \(table_m\). Then it will check if there is any composite event \(e_{comp}\) which uses \(e_n\) and another event \(e_j\) as its sub-event (\(e_{comp}=comp(e_n, e_j)\)). The pseudo code of event matching is shown in Algorithm \ref{algo:eventmatching}.

Note that if \(e_{comp}\) has been successfully detected, the procedure will run function \(detected\) which, in turn, will execute Algorithm \ref{algo:eventforwarding}. The procedure will also use \(event\_forward\) in the end to forward the detected event to other fusion node so that better fusion point may be found.

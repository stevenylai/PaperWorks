\section{Clustering Algorithms}
In this section, we will formulate the optimal clustering problem and then describe the corresponding solutions to the problem.

\subsection{Problem Formulation}
\label{sec:OptimalClustering}
The objective of clustering is that the generated clusters can minimize the energy consumption of overall modal analysis. Clustering also has to satisfy the following constraints (1) each sensor node belongs to at least one of the generated clusters, (2) sensor nodes in each cluster is within a single communication range to its CH, (3) number of sensor nodes in each cluster is larger than \(p\) (\(p\):the number of mode shape vectors to be identified) (4) all the clusters are connected together through the overlapping nodes. More formally, problem is formulated as follows:
Given a sensor network \(G=(V,E)\), find a clustering scheme that can cluster these \(V\) sensor nodes into a set of clusters, denoted as \(C=\{S_1, S_2, S_3, \cdots\}\), subject to the following constraints:
\begin{enumerate}
	\item \( \bigcup\limits_{S_i \in C} = V\)
	\item Let the sub-graph for \(S_i\) is \(G(S_i,E_i)\), where \(E_i \subseteq E\). Then \(\forall S_i \in C, \exists s_i \in S_i\), such that there is an edge \(a_{ij} \in E_i\) between \(s_i\) and any other \(s_j \in S_i (s_i \neq s_j)\) 
	\item \(\forall S_i \in C, \left|S_i\right|\geq p\)
	\item \( \forall S_i, \exists S_j \in C, (i \neq j), S_i \bigcap S_j \neq \emptyset\)
	\item \(\forall C' \subseteq C, (\bigcup\limits_{S_i \in C'} S_i)\bigcap (\bigcup\limits_{S_j \in C-C'} S_j) \neq \emptyset \)
\end{enumerate}
Objective:
\begin{itemize}
\item Minimize \(\sum\limits_{S_i \in C} cost(S_i)\)
\end{itemize}

The first constraint is set because we wish to find the mode shapes defined on all the deployed sensor nodes. The second constraint is to ensure only single-hop clusters are generated. Constraint 3 is required by the ERA algorithm. Constraints 4 and 5 are used to describe that generated clusters are overlapping and connected. 

The above clustering problem is an optimization problem. We will prove that the decision version of the problem is NP complete which is defined as: \emph{given a threshold \(k\) , does there exist a cluster set \(C=\{S_1, S_2, S_3, \cdots\}\), which satisfy all the constraints above and whose total energy cost \(\sum\limits_{S_i \in C} cost(S_i)\) is equal of smaller than \(k\)?}.

\begin{theorem}
The decision version of our clustering problem is NP-complete.
\end{theorem}

\begin{proof}
It is easy to find out this clustering problem is NP.  Given a cluster set \(C\), all the constraints above, include constraints 4 and 5 can be checked in a polynomial time. The detailed proof of this part is omitted for brevity.

We show this decision version of the clustering problem is NP-hard by reducing the set cover problem to it. The set cover problem is defined as follows.

\begin{flushleft}
\textbf{Given:}
\end{flushleft}
\begin{enumerate}
\item A universe \(V'\)
\item A set of \(S'=\{S'_1, S'_2, ...\}\subseteq V'\)
\item The cost function for each subset \(S'_i\in S'\): \(cost'(S'_i)\) 
\item A number \(k'\)
\end{enumerate}
\textbf{Find: }
If there is a subset \(C'\subseteq S'\) which satisfies
\begin{enumerate}
\item \( \bigcup\limits_{S'_i \in C'} S'_i =V'\)
\item \( \sum\limits_{S'_i \in C'} cost'(S'_i) \leq k'\)
\end{enumerate}

To reduce the set cover problem to the clustering problem, we construct a sensor network \(G=(V, E)\) from the inputs of set cover problem in the following way:
The vertices \(V =V'\bigcup X\) , where \(X = \{x_1,x_2,...x_p\}\)  is a set of \(p\) virtual nodes. 
To construct the edges \(E\), for each \(S'_i\in S'\) , we first choose an arbitrary  node \(s'_i\in S'_i\), then we add an edge between \(s'_i\) and any other node \(s'_j \in S'_i (s'_j\neq s'_i)\). We also add an edge between \(s'_i\) and any virtual node in \(X\). 
The cost function in the clustering problem \(cost(\cdot) = cost'(\cdot)\). We also define that by adding/deleting any virtual node \(x\) to/from any group will not affect the cost function. The energy threshold \(k=k'\). 

With this transformation, it can be easily proved that 1) Assume \(C'=\{S'_1, S'_2, ...\}\)is a solution to the set cover problem, then \(C=\{S_1, S_2, ...\}\)  is a solution to the clustering problem, where \(S_i = S'_i \bigcup X\). 2) Assume \(G =(V,E)\) is constructed from the set cover problem and we have a solution \(C=\{S_1, S_2, ...\}\) to the clustering problem, then \(C'=\{S'_1, S'_2, ...\}\) is a solution to the set cover problem, where \(S'_i=S_i - X\). The detailed proof is omitted for brevity. 
\end{proof}
By reducing the NP-complete set problem to our clustering problem, we have demonstrated that the decision version of our problem is NP-complete. Obviously, the original clustering problem is also NP-Complete.
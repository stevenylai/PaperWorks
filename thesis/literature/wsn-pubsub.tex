\section{Related Works}
There are limited amount of works on pub/sub based middleware for WSN. However, works in areas such as middleware for WSN, programming abstractions for WSN and data gathering techniques in WSN are closely related to our work. In this section, we will study the representative works in those related fields and summarize them.
\subsection{Pub/Sub in WSN}
Up till now, there are quite few works on designing pub/sub systems in WSN and all of them support only simple event subscription. In this section, we review and summarize three most important works.

Low-level naming \cite{lowlevelnaming}: this is a pub/sub system based on Directed Diffusion \cite{directeddiffusion}. In this system, the authors mainly focus on discussion of attribute-based communication. Based on the attribute-based naming mechanism, the authors discuss the matching rules and the filters. The matching rules are largely based on the previous work on Linda \cite{linda} where actual and formal are used. More specifically, actual is the value of a given attribute and formal can be regarded as the operator or the wildcard of the values. The authors also propose the idea of filters which are the software modules that will be executed once triggered by matching rules. Based on these ideas the authors implement a pub/sub system for sensor network where sensor nodes can subscribe to interesting attributes.

Diffusion filters \cite{diffusionfilters}: this work is from the same group of authors of 'low-level naming'. In this paper, the authors have a further discussion on the filters. A few examples are given. For example, the gradient filter is used to send reinforcement. The filters can also be used to implement GEAR \cite{gear} (Geographic and Energy-Aware Routing) which is a geographic routing protocol implemented by two filters.

Mires \cite{mires}: this work is just a repack of the 'Surge' application in TinyOS. The system is built on two routing protocols (so as 'Surge'). Bcast is used to broadcast a message from sink to the entire network. MultihopRoute is used to let each sensor node transmit its data to the sink. In this system, the authors first use MultihopRoute to let each sensor node send its own advertisement to the sink. Then the sink use Bcast to broadcast the subscription. After that, each sensor node uses MultihopRoute to deliver the events. The core idea of Mires is actually very similar to low-level naming in the sense that subscription is propagated by broadcasting and events are published based on the shortest path found by the underlying routing protocols.

Semi-probabilistic Approach \cite{sp}: this paper proposes an idea that trade-off reliability with energy consumption. The basic idea is similar to directed diffusion but with a probabilistic model. Instead of propagate the subscription into the whole network, the authors choose to stop forwarding the subscription after a certain number of hops. The publisher will then deliver the events in an opportunistic way. If a node with subscription information is reached, the events will be delivered. Otherwise, the events will be forwarded with a certain probability.

Summary: there aren't many works done that implement pub/sub systems in WSN and most of the existing works are similar to each other so there is still a lot of room for doing research in this area.


\section{A Centralized Approach}
\label{sec:centralized}
Based on our problem formulation, we can intuitively use a centralized approach to solve this problem. In the centralized approach, when the sensor nodes detect the events, they first inform the sink node about the detected event. Then the sink node will select the efficient nodes as event fusion points for each event type and broadcast the information into the network. The nodes will then forward their events to these fusion points. The fusion points may be changed or updated with a predetermined probability to cope with event dynamics. The reasons why the nodes will only inform the sink about the events they detected rather than send the events to the sink for detection are as follows:
\begin{itemize}
\item Each individual event may have a lot of attributes so it is more desirable to first calculate the energy-efficient event fusion points before the actual event detection.
\item As described in Section \ref{sec:introduction}, events may have strong locality. Therefore, the fusion points selected may be used for detecting events of the same types in the future.
\end{itemize}

We assume the users have no prior knowledge on where the events might happen. However, they do know the probability distribution of the events. We will discuss the event probability distribution and its impact on the algorithm in the latter section when we describe how the algorithm is designed to cope with event dynamics.

The input of the algorithm includes:
\begin{itemize}
\item Sensor network: \(G\)
\item A set of events \(E'=\{e_i^1, e_i^2, \cdots \}\) where \(e_i^j\) denotes an event of type \(e_i\)
\end{itemize}

\begin{algorithm}
\begin{algorithmic}[1]
\REQUIRE \(G=(N,A)\), \(E'\)
	\FORALL {\(e_i^j\in E'\)}
		\IF {\(e_i\) is not assigned to any node}
			\STATE Find any other \(e_k^l\in E'\) such that either k==i or \(e_k\) and \(e_i\) has a relation for a composite event \(e_x\)
			\FORALL {\(n \in N\)}
				\STATE Construct an SPT for all \(e_k^l\) and \(e_i^j\) with root as \(n\)
				\STATE Calculate the cost for the SPT as \(cost_n\)
			\ENDFOR
			\STATE Find the SPT with the smallest \(cost_n\) and select \(n\) as fusion point for \(e_k\) and \(e_i\)
			\STATE Add \(n\) as the node detecting \(e_x^1\) to \(E'\)
		\ENDIF
	\ENDFOR
\end{algorithmic}
\caption{Centralized TED}
\label{algo:centralizedTED}
\end{algorithm}

The algorithm will examine all the primitive events reported from the network to see if there is any relation associated with them. If so, the algorithm will try each node in \(G\) to construct shortest path tree (SPT) for these related events and select the one with the smallest cost. The root of that SPT will then become the fusion point for the related events. The algorithm runs recursively in the sense that once the fusion points for a composite event is selected, itself will become the node for detecting that composite event.

\subsection{Determine the Re-selection Probability}
While the previous section outlines the algorithm for centralized TED, we still need to decide how often the nodes should switch to another fusion point in order to cope with the event dynamics. We use exponential distribution as the event probability distribution because of its memoryless property. More specifically, for each composite event, the distance \(x\) between each of its sub-events to any point in the network follows an exponential distribution as follows:
\begin{displaymath}
f(x)={\lambda}_1e^{-{\lambda}_1x}
\end{displaymath}

In addition to the distance between events, the direction of events that happen in different rounds will also affect the selection probability. The angle \(\theta\) between any pair of related events also satisfy an exponential distribution as follows:
\begin{displaymath}
f(\theta)={\lambda}_2e^{-{\lambda}_2\theta}
\end{displaymath}

Both \(x\) and \(\theta\) are shown in Figure \ref{fig:fp-dist} where the events \(e_1\), \(e_2\) are detected before and \(e'_1\), \(e'_2\) are detected. For simplicity, we use distance to measure the cost for one node to reach another. Let \(d\) be the average distance between any point in the deployment region to the closest sensor node. As shown in Figure \ref{fig:fp-dist}, for a composite event that has two sub-events, originally the event is fused at \(n_1\). Then, for the next detection of \(e'_1\) and \(e'_2\), if the fusion point is still the original one, then the cost will be the added distance of \((e'_1, n_1)\) and \((e'_2, n_1)\):
\begin{align*}
cost_1=\sqrt{({\lambda}_1)^2+(\frac{{\lambda}_1}{2})^2+({\lambda}_1)^2cos(\frac{\pi+{\lambda}_2}{2})}
\end{align*}
On the other hand, if a new fusion point is selected, then the cost will be no more than:
\begin{align*}
cost_2=&2(\frac{{\lambda}_1}{2sin\frac{{\lambda}_2}{2}}+{\lambda}_1)sin\frac{{\lambda}_2}{2}+d\\
=&{\lambda}_1(1+2sin\frac{{\lambda}_2}{2})+d
\end{align*}

\begin{figure}
\centering
\subfloat[Event distribution]{\label{fig:event-distribution1}\figurehalfwidth{event-distribution1}}
\subfloat[Fusion point distance]{\label{fig:fp-dist}\figurehalfwidth{fp-dist}}
\caption{Selecting fusion points in centralized approach}
\label{fig:centralizedTED}
\end{figure}

The condition to select a new fusion point will be:
\begin{align}
&cost_1\geq cost_2\nonumber \\
&cost_1^2\geq cost_2^2\nonumber \\
&({\lambda}_1)^2(\frac{5}{4}+cos(\frac{\pi+{\lambda}_2}{2}))\geq\nonumber \\ &{\lambda}_1^2(1+2sin\frac{{\lambda}_2}{2})^2+d^2+2{\lambda}_1(1+2sin\frac{{\lambda}_2}{2})d
\label{eq:fpSwitch}
\end{align}

Here \(d\) is decided by the node density and can be estimated once we know the deployment area and the number of nodes in the deployment area. Therefore, given \(f(x)\), \(f(\theta)\) and \(d\), we can use generalized gradient search to find the values for \(\theta\) and \(x\) such that Equation \ref{eq:fpSwitch} is satisfied while minimizing the probability for switching:
\begin{align*}
P_{switch}&=F(x>x*)F(\theta>\theta *)\\
&=\int_{x*}^{\infty}{\lambda}_1e^{-{\lambda}_1x}dx\int_{\theta *}^{\infty}{\lambda}_2e^{-{\lambda}_2\theta}d\theta\\
&=e^{-{\lambda}_1x*}e^{-{\lambda}_2\theta *}
\end{align*}

Once the probability is obtained, after each event detection, there will be a probability of \(P_{switch}\) that the event will switch to another fusion point. This is done by making the fusion point broadcast a message in the network so that all nodes can delete the corresponding event type assignment to that fusion point.

So far the calculation for \(P_{switch}\) is based on the assumption that the network scale is not known in advance and the cost for broadcasting new fusion point information is not known in advance. If the network scale is known to be \(|N|\) in advance, then we can change Equation \ref{eq:fpSwitch} a bit to be more accurate as in Equation \ref{eq:fpSwitchAccruate}.

\begin{align}
cost_1\geq cost_2+|N|\times P_{switch}
\label{eq:fpSwitchAccruate}
\end{align}

By calculating \(|N|\times P_{switch}\), we can further fine-tune the switching probability to balance with the broadcast cost.
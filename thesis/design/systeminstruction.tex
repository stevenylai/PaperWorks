\section{Complete List of the Instructions for EDF}
\label{appendix:isa}
In this section, we show the complete list of instructions currently used by our Event Detection Framework (EDF). We also briefly describe the function and the usage of each instruction.

\subsection{Basic Instructions}
The basic instruction deals with the most fundamental operations. Since the instruction uses a stack-based architecture in order to reduce the code size, most of the instructions that fall into this category deal with the operations related stacks.
\begin{itemize}
\item \emph{OPpush}: this instruction is used to push an operand to the top of the stack.
\item \emph{OPpop}: this instruction is used to pop and discard an operand from the top of the stack.
\item \emph{OPcopy}: duplicates the top operand of the stack
\item \emph{OPhalt}: once this instruction is reached, the VM will stop execution.
\end{itemize}

\subsection{Operators}
Basic mathematical operators:
\begin{itemize}
\item \emph{OPadd}: pops two operands off the top of the stack, add them together and push the result back to the top of the stack.
\item \emph{OPmult}: pops two operands off the top of the stack, multiply them together and push the result back to the top of the stack.
\item \emph{OPsub}: pops two operands off the top of the stack, subtract the first popped number by the second one and push the result back to the top of the stack.
\item \emph{OPdiv}: pops two operands off the top of the stack, divide the first popped number by the second one and push the result back to the top of the stack.
\item \emph{OPmod}: pops two operands off the top of the stack, divide the first popped number by the second one and push the remainder back to the top of the stack.
\item \emph{OPinv}: pops one operand (n) off the top of the stack, calculate its inverse (-n) and push the result back.
\end{itemize}

Logical operators:
\begin{itemize}
\item \emph{OPand}: pops two operands off the top of the stack, calculate the logical and of them and push the result back to the top of the stack.
\item \emph{OPor}: pops two operands off the top of the stack, calculate the logical or of them and push the result back to the top of the stack.
\item \emph{OPxor}: pops two operands off the top of the stack, calculate the exclusive or of them and push the result back to the top of the stack.
\item \emph{OPnot}: pops one operand off the top of the stack, calculate the logical not of it and push the result back to the top of the stack.
\end{itemize}

Relational operators:
\begin{itemize}
\item \emph{OPeq}: pops two operands off the top of the stack, if the two operands are equal, push 1 to stack. Otherwise, push 0 to stack.
\item \emph{OPneq}: pops two operands off the top of the stack, if the two operands are equal, push 0 to stack. Otherwise, push 1 to stack.
\item \emph{OPgt}: pops two operands off the top of the stack, if the first operand is greater than the second one, push 0 to stack. Otherwise, push 1 to stack.
\item \emph{OPgte}: pops two operands off the top of the stack, if the first operand is greater than or equal to the second one, push 0 to stack. Otherwise, push 1 to stack.
\item \emph{OPlt}: pops two operands off the top of the stack, if the first operand is less than the second one, push 0 to stack. Otherwise, push 1 to stack.
\item \emph{OPlte}: pops two operands off the top of the stack, if the first operand is less than or equal to the second one, push 0 to stack. Otherwise, push 1 to stack.
\end{itemize}

Bitwise operators:
\begin{itemize}
\item \emph{OPshiftl}: pops two operands off the top of the stack, left shift the first popped number by the second one and push the result back to the top of the stack.
\item \emph{OPshiftr}: pops two operands off the top of the stack, right shift the first popped number by the second one and push the result back to the top of the stack.
\item \emph{OPland}: pops two operands off the top of the stack, calculate the bitwise and of them and push the result back to the top of the stack.
\item \emph{OPlor}: pops two operands off the top of the stack, calculate the bitwise or of them and push the result back to the top of the stack.
\item \emph{OPxor}: pops two operands off the top of the stack, calculate the bitwise exclusive or of them and push the result back to the top of the stack.
\item \emph{OPlnot}: pops one operand off the top of the stack, calculate its complement and push the result back to the top of the stack.
\end{itemize}

\subsection{Event-related Instructions}
\begin{itemize}
\item \emph{OPref}: whenever an event \(e\) is being evaluated, this instruction is invoked to obtain an instance of such event.
\item \emph{OPoffset}: this instruction is involved right after the 'ref' instruction, in order to access individual attributes of the event instance.
\item \emph{OPset}: if the attributes of an event need to be changed, this instruction will be used.
\item \emph{OPget}: the 'get' instruction does the opposite of 'set' instruction. It will simply retrieve content of a specific attribute in an event.
\item \emph{OPcreate}: is used to create a new instance of an event.
\item \emph{OPeval}: is used to determine if an event happens or not.
\item \emph{OPgc}: used by the event matcher for garbage collection.
\end{itemize}
\begin{abstract}
Event detection is an important topic in many WSN applications. Although there are several works on providing event-based services in WSN, most of them can only deal with primitive event types but cannot handle composite events very well. In general, a type-based event system may have primitive or composite event types. All event types are defined by specifying their attributes and filters. Then, individual events are detected and delivered according to such type information. Composite event types may be defined by combining multiple event types with operators. Due to the resource constraints in WSN, composite events are much more difficult to manage than primitive events. In this work, we introduce PSWare, a type-based publish / subscribe middleware for WSN that supports composite events. We describe our design for PSWare. PSWare has a flexible architecture where different composite event detection algorithms may be easily integrated.

On top of PSWare, we present TED (Type-based Event Detection), a novel distributed composite event detection algorithm. The essential idea of TED is event fusion, where some sensor nodes are selected as fusion points and component events are fused for the detection of a high level event. Event fusion with minimum energy cost is an NP-complete problem. Therefore, TED uses a number of heuristics with bounded performance.  

To further increase the performance of event detection, we design a clustering algorithm for PSWare for structural health monitoring applications (SHM). We formulate the problem and found it to be NP-complete so we propose heuristic centralized and distributed algorithms. In addition to SHM, We use PSWare to develop other real world WSN applications including intelligent transportation system (ITS) and in-door monitoring.

We evaluate our system from different aspects. We first evaluate TED, the most essential algorithm in our system. We analyze its performance and then carry out extensive simulations. Both analytical and simulation results show TED can save energy in event-based applications where primitive events occur in a higher frequency than composite events. Then we combine TED with PSWare and carried out some real world experiments. The results show that PSWare can offer reasonably simple API for the application developers to use while TED and our clustering algorithm can improve the underlying event detection performance. Compared with opportunistic approaches to event detection in these real applications, PSWare can reduce 40 - 50 \% of the energy cost.
\end{abstract}
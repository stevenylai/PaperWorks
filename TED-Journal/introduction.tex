\section{Introduction}
\label{sec:introduction}
Composite event \cite{jector} detection in WSNs is required in many applications such as health care \cite{lo:ban}, smart building \cite{lynch:shm} and intelligent transportation system \cite{klein:its}. For instance, in an intelligent transportation system, we may define traffic jam as an event when there are certain number of cars waiting on a road. Such kind of event may come from many sub-events such as the number of the vehicles on a road and the speed of the vehicles. Furthermore, the speed of the vehicles may come from the events of each individual vehicle. All these sub-events must satisfy certain spatial or temporal relations in order to indicate an occurrence of a composite event.

In general, composite event detection in WSN can be considered as a special type of predicate detection \cite{gag:predicate} and events in WSN applications have two special properties. First, events in many WSN applications have short life spans. For instance, object tracking usually involve events that occur as the object passes the network. After the object has exited the monitoring region, the primitive events related to that object will usually become useless. Second, events in WSN may have strong locality.  For example, consider a surveillance application where the events are defined to characterize intrusions into certain region, such events usually happen at the the entrance or the exit of the region. 

Moreover, energy is usually considered as a crucial issue for WSN. It is more preferable to detect composite events in-network instead of gathering all the primitive events at sink and performing centralized event detection. As a result, event detection in WSN quite different from event detection in active databases \cite{samos} when issues such as network dynamics and resource constraints are considered.

As an important paradigm for distributed communication, pub/sub is especially suitable for event-based applications due to its decoupling \cite{facespubsub}. There are mainly three kinds subscriptions used in pub/sub system: topic-based subscription, content-based subscription and type-based subscription. In particular, composite events are usually defined in type-based pub/sub system by specifying event types and their relations. Certain existing works \cite{lai:psware} have addressed the issue of defining and subscribing such kind of composite events for WSN. However, how to perform energy efficient composite event detection has not been fully addressed in these works.

In this paper, we formally address the problem of composite event detection for WSN. We propose TED (Type-based Event Detection), a distributed event detection algorithm to the problem. The main idea behind TED is to detect composite events based on their types. Then assign each composite event type to a certain node called event fusion point so that the composite events can be efficiently detected. TED uses both a centralized and a distributed scheme for selecting fusion points. The contribution of this paper can be summarized as follows:
\begin{enumerate}
  \item We formulate the problem of type-based composite event detection and prove it NP-complete.
  \item We propose TED, a distributed type based event detection algorithm. There are two version of TED, one with a centralized fusion point selection scheme and the other using distributed fusion point selection scheme.
  \item We validated our algorithms through analysis, simulations and experiments. The results show the energy efficiency of TED in event-based WSN applications.
\end{enumerate}

The rest of the paper is organized as follows: Section \ref{sec:background} presents the background for our composite event detection approach. Section \ref{sec:relatedworks} reviews related works. Section \ref{sec:system_model} presents our system model and problem formulation. Section \ref{sec:centralized} presents TED with centralized fusion point selection. Section \ref{sec:cedu} describes fully distributed TED in details. Section \ref{sec:ceduevaluation} shows the performance of TED through analysis, simulation and experiments. Section \ref{sec:conclusion} concludes the paper.


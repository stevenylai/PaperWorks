\section{Related Works}
\label{sec:relatedworks}
Existing works on pub/sub systems for WSN primarily consider the detection of primitive events where each event is usually treated separately and events don't have relations between each other. \cite{lowlevelnaming} is a pub/sub system built on top of directed diffusion \cite{directeddiffusion}. The sink node will first broadcast interest and the source nodes will deliver the detected events through gradients and reinforced paths. Mires \cite{mires} is a pub/sub middleware for WSN. It makes use of the message-oriented communication paradigm provided by TinyOS \cite{nesc}. First, nodes will advertise their available topics using a multi-hop routing protocol. Then, the sink will broadcast the subscription and finally, nodes will be able to publish the events to the sink. \cite{sp} had an in-depth discussion on the trade-off between reliability and energy consumption. Instead of flooding the subscription to every node in the network, the flood stops after certain number of hops. If the subscription does not reach the publisher, then the event will be forwarded probabilistically. 

More recently, certain works such as \cite{lai:ted, complexevent} have been proposed for composite event definition and detection. The primary focus of \cite{lai:ted} is on a composite event detection algorithm called TED which utilizes event type information for efficient detection. In addition to an event language, \cite{complexevent} also discusses how to reliably detect composite events in a pervasive environment. The focus of this paper is on a flexible middleware framework so that different event detection algorithms may be integrated and evaluated easily.

Apart from the pub/sub paradigm, there have been a lot of efforts on developing other type of programming abstractions for WSN including query-based approaches \cite{cougar, sina, tinydb} VM-based approaches \cite{magnetos, mate, smartmessage} tuple space-base approaches \cite{tinylime} neighbor-based approach \cite{kairos, hood, abstractregion} and mobile agent-based approaches \cite{agilla, sensorware}.

In terms of composite event detection in WSN, there have been very limited works. However, a lot of work has been done on data aggregation for sensor network. Existing data aggregation can be mainly divided into three categories: cluster-based approach \cite{leach, iheed, epas}, chain-based approach \cite{pegasis} and tree-based approach \cite{mfst, dctc, tag, xue:lp, tina}. Cluster-based approach typically considers the problem how to select and rotate cluster heads so that the clusters can be evenly distributed in the network \cite{iheed} and energy consumption will be balanced \cite{leach}. Cluster-based approach can be organized into multiple levels in order to further save the cost. Chain-based approach improves cluster-based approach by letting each sensor node only communicate with its close neighbors \cite{pegasis}. Tree-based approaches have many differnt optimizatin techniques. For example, \cite{xue:lp} formulates the problem as a multi-commodity flow problem and uses linear programming to solve it.  MFST \cite{mfst} constructs a minimum Steiner tree with a cost model that considers the fusion cost. 

We believe the pub/sub paradigm is suitable for many WSN applications because many of these applications are event-based in nature. However, more work needs to be done in order to support efficient composite event detection, especially how to support application-specific event detection mechanisms so that high energy efficiency can be achieved.